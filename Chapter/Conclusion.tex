\chapter{Conclusion}
\label{chap:conclusion}
This thesis presents a novel approach to creating photorealistic 3D virtual environments for robotics, leveraging the synergy between Luma AI’s Neural Radiance Fields (NeRF) and Unreal Engine 5 (UE5). The research successfully demonstrates a cost-effective and efficient pipeline for generating these environments using standard smartphone cameras, significantly reducing the time and resources compared to existing methods like NeRF2real and Matterport3D. The practical application of this approach is evidenced by the deployment of the Turtlebot3 robot in these virtual environments, controlled via ROS2 Humble, and the achievement of high F1 scores in cross-environment object recognition. This methodology facilitates more effective simulation-driven development and testing in robotics, setting a new standard for virtual environment generation. However, the reliance on Luma AI as a sole processing platform is identified as a potential limitation, indicating a direction for future research towards developing independent or alternative platforms. The thesis contributes significantly to the field of robotic simulation, offering a practical solution to the challenges of creating and utilizing photorealistic 3D environments in robotics.





