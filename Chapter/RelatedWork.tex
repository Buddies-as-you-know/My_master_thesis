\chapter{Related Work}
\label{cha:Related Work}
\section{Photorealistic scene reconstruction technology}
\label{sec:Photorealistic scene reconstruction technology}

Our approach to indoor environment modeling differs from traditional datasets. While common datasets like Matterport3D\cite{chang2017matterport3d} and Gibson\cite{xia2018gibson}, Replica\cite{straub2019replica} use custom-built scanning equipment with depth and LIDAR capabilities, we use a small number of video images from consumer-grade mobile cameras to train NeRF and model the scene. This approach is fundamentally different from other studies, such as Habitat-Matterport3D\cite{shah2018airsim}.


\section{Neural Radiance Fields in robotics}
\label{sec:Neural Radiance Fields in robotics}

NeRF has a wide range of applications in robotics. However, our research takes a different approach from previous work. NeRF has been used for posture estimation\cite{lin2021inerf}, representation learning\cite{lin2022nerf-supervision}, grasping behavior\cite{ichnowski2021dex-nerf}, and learning dynamic models\cite{driess2022learning}. Although traditional state estimation and planning methods for obstacle avoidance in simulated NeRF environments have been studied, their real-world applications remain untested. Our research will focus on creating a simulation environment for these studies.

\section{Unreal engine in robotics}
\label{sec:Unreal engine in robotics}
The Unreal Engine has been widely used in robotics research, especially in developing simulation environments. This emphasizes the engine's capability to produce immersive and authentic virtual environments for off-road autonomous driving and robotic vision \cite{young2020unreal, martinez-gonzalez2019unrealrox}. It showcases the engine's potential in creating accurate ground-truth datasets and real-time physics simulations of robotic behavior \cite{pollok2019unrealgt}. These studies highlight the Unreal Engine's adaptability and efficiency in driving forward research in robotics.