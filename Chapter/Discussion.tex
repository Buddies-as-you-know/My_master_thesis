\chapter{Discussion}
\label{chap:discussion}

\section{Virtual Environment Creation}
\label{sec:Virtual Environment Creation}
The results pertaining to virtual environment creation (Section \ref{sec:ComparisonofComputationalEfficiencyandResourceUsage}) underscore a significant disparity in the balance between ease of use, resource requirements, and technical expertise among the compared methods. It is evident that while methods like NeRF2Real and Matterport3D offer high precision and quality in environment creation, they necessitate substantial technical expertise and resources, thus limiting their accessibility to specialists and well-equipped organizations. In contrast, Our Method, with its minimal resource requirement and elimination of the need for specialized expertise, democratizes the process of virtual environment creation. This accessibility suggests a potential shift in the field towards more user-friendly and resource-efficient methodologies, particularly beneficial for individuals and smaller entities. The implications here are twofold: firstly, the democratization of virtual environment creation could lead to broader adoption and innovation in various fields, and secondly, it raises questions about the balance between quality and accessibility in technological advancements.

\section{Object Recognition Performance}
\label{sec:Object Recognition Performance}
The study of object recognition performance in real and virtual environments (Section \ref{sec:YoloResult}) reveals significant findings regarding the efficacy of virtual environments in replicating real-world scenarios. The high agreement rate of over 80\% in object recognition models across these environments suggests that virtual environments can effectively mirror the real world. However, the variation in performance, as observed in the kiosk and research building environments, indicates the influence of environmental complexity and object diversity on model accuracy. This underlines the importance of considering environmental factors when deploying object recognition models. The successful application of the YOLO model further validates the reliability of virtual environments in simulating real-world settings, which can have profound implications for the fields of robotics and AI training.

\section{Navigation Technology in Virtual Environments}
\label{sec:Navigation Technology in Virtual Environments}
The exploration of Navigation 2 technology, as discussed in Section \ref{sec:Navigation2Result}, represents a significant stride in the application of virtual environments for navigation system development. The successful simulation of navigation within a virtual environment using combined technologies from LumaAI and Unreal Engine 5 highlights the potential of virtual environments as a testing ground for developing and refining navigation technologies. This has crucial implications for areas such as robotics, autonomous vehicles, and AI, where real-world testing can be resource-intensive and potentially hazardous.

\section{Limitations and Future Directions}
\label{sec:Limitations and Future Directions}
While the study provides valuable insights, there are limitations that need to be addressed in future research. One major limitation is the scope of environments and models tested. Extending the research to include a broader range of environments and object recognition models would provide a more comprehensive understanding of the capabilities and limitations of virtual environments. Additionally, exploring the scalability of these technologies in more complex and dynamic real-world scenarios would provide further validation of their practicality and effectiveness.

This study contributes significantly to the understanding of virtual environment creation, object recognition in different settings, and navigation technologies. The findings suggest a shift towards more accessible and resource-efficient technologies in virtual environment creation, affirm the effectiveness of virtual environments in replicating real-world scenarios for object recognition, and highlight the potential of virtual settings in developing navigation technologies. However, future research addressing the noted limitations is essential to further advance the field.





